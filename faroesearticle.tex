%!TEX program = xelatex

\documentclass[a4paper]{article}

\usepackage{fontspec}
   \defaultfontfeatures{Ligatures=TeX}
   \setmainfont{Linux Libertine}
   \setsansfont[Scale=MatchLowercase]{Fira Sans}
   \setmonofont[Scale=MatchLowercase]{Fira Mono}
\usepackage[margin=1in]{geometry}
\usepackage{microtype}
\usepackage[backend=biber,style=unified,maxnames=99]{biblatex}
   \addbibresource{../bibliography/faroeserefs.bib}
\usepackage{graphicx}
\usepackage{expex}
\usepackage{sectsty}
   \sectionfont{\large}
\usepackage{datetime2}
\usepackage{textglos}
\usepackage{booktabs}
\usepackage[hang,flushmargin]{footmisc}
\usepackage[dvipsnames]{xcolor}
\usepackage[colorlinks]{hyperref}
   \hypersetup{allcolors=black,urlcolor=MidnightBlue}

\newcommand*{\ipa}{\textipa}

\title{Phonological variation in Faroese}
\author{Alec Shaw}
\date{\today}

\begin{document}

\maketitle

\section{Introduction}

\ex Monophthongs \parencite[68]{arnason:11a}\footnote{\textsc{fem} = feminine; \textsc{masc} = masculine; \textsc{neut} = neuter} \\[1ex]
   \vtop{\halign{%
      #\hfil & #\hfil & \quad #\hfil \cr
      \xm{i} & \xv{linur} \xt{\ipa{li:nU\*r}} \gl{soft} & \xv{lint} \xt{\ipa{lI\r*nt}} \gl{soft-{\sc nom}} \cr
      \xm{e} & \xv{frekur} \xt{\ipa{f\*re:(\super{h})kU\*r}}/\xt{\ipa{freE:(\super{h})kU\*r}} \gl{greedy} & \xv{frekt} \xt{\ipa{frE\super{h}kt}} \gl{greedy-{\sc nom}} \cr
      \xm{y} & \xv{mytisk} \xt{\ipa{my:tIsk}} \gl{mythological} & \xv{mystisk} \xt{\ipa{mYstIsk}} \gl{mysterious} \cr
      \xm{ø} & \xv{høgur} \xt{\ipa{hø:VU\*r}}/\xt{\ipa{hø{\oe}VU\*r}} \gl{high-{\sc masc}} & \xv{høgt} \xt{\ipa{h{\oe}kt}} \gl{high-{\sc neut}} \cr
      \xm{u} & \xv{gulur} \xt{\ipa{ku:lU\*r}} \gl{yellow} & \xv{gult} \xt{\ipa{kU\r*lt}} \gl{yellow-{\sc neut}} \cr
      \xm{o} & \xv{tola} \xt{\ipa{t\super{h}o:la}}/\xt{\ipa{t\super{h}oO:la}} \gl{to endure} & \xv{toldi} \xt{\ipa{t\super{h}Ol\r*dI}} \gl{endured} \cr
      \xm{a} & \xv{Kanada} \xt{\ipa{k\super{h}a:nata}} \gl{Canada} & \xv{land} \xt{\ipa{lant}} \gl{land} \cr
      \cr
   }}

   Diphthongs \\[1ex]
   \vtop{\halign{%
      #\hfil & #\hfil & \quad #\hfil \cr
      \xm{\ipa{Ui}} & \xv{hvítur} \xt{\ipa{kfUi:tU\*r}} \gl{white-{\sc masc}} & \xv{hvítt} \xt{\ipa{kfUi\super{h}t:}} \gl{white-{\sc neut}} \cr
      \xm{\ipa{Ei}} & \xv{deyður} \xt{\ipa{tei:jU\*r}} \gl{dead-{\sc masc}} & \xv{deytt} \xt{\ipa{tE\super{h}t:}} \gl{dead-{\sc neut}} \cr
      \xm{ai} & \xv{feitur} \xt{\ipa{fai:tU\*r}}/\xt{\ipa{fOi:tU\*r}} \gl{fat-{\sc masc}} & \xv{feitt} \xt{\ipa{fai\super{h}t:}}/\xt{\ipa{fOi\super{h}t:}} \gl{fat-{\sc masc}} \cr
      \xm{\ipa{Oi}} & \xv{gloyma} \xt{\ipa{klOi:ma}} \gl{to forget} & \xv{gloymdi} \xt{\ipa{klOimtI}} \gl{forgot} \cr
      \xm{\ipa{Ea}} & \xv{spakur} \xt{\ipa{spEa:(\super{h})kU\*r}} \gl{calm-{\sc masc}} & \xv{spakt} \xt{\ipa{spakt}} \gl{calm-{\sc neut}} \cr
      \xm{\ipa{Oa}} & \xv{vátur} \xt{\ipa{vOa:tU\*r}}/\xt{\ipa{vA:tU\*r}} \gl{wet-{\sc masc}} & \xv{vátt} \xt{\ipa{vO\super{h}t:}} \gl{wet-{\sc neut}} \cr
      \xm{\ipa{0u}} & \xv{fúlur} \xt{\ipa{f0u:lU\*r}}/\xt{\ipa{fIu:lU\*r}} \gl{foul-{\sc masc}} & \xv{fúlt} \xt{\ipa{fY\r*lt}} \gl{foul-{\sc masc}} \cr
      \xm{\ipa{Ou}} & \xv{tómur} \xt{\ipa{t\super{h}Ou:mU\*r}}/\xt{\ipa{t\super{h}{\oe}u:mU\*r}} \gl{empty-{\sc masc}} & \xv{tómt} \xt{\ipa{t\super{h}{\oe}\r*mt}}/\xt{\ipa{t\super{h}O\r*mt}} \gl{empty-{\sc neut}} \cr
   }}
\xe

\section{Vowel reduction in restricted syllables}

\section{Raising in hiatus}

\section{Conclusion}

\appendix
\section{Excerpt from \textit{Hestsøgu}: \textit{Mýs} \parencite[177]{lockwood:02}}\label{app:readingtext}

Mýs hava ongar verið í Hesti frá gamal tið av. Men o. u.\footnote{o.u. = okkurt um.} árið 1908 varð Niklas á Bakka varur við, at hann hevði fingið mýs í húsið, og ikki vardi leingi, fyrr enn tær vóru um alla bygdina. Beinan vegin varð farið undir at royna at fáa ruddað tær út við fellum. Kettur vildi eingin fáa sær, tí allir meintu, at tær komu at styggja mýsnar burtur frá húsunum, og so varð vónleyst at fáa ruddað tær út. Fellur vórðu settar allastaðni og helst í hvørjum úthúsi. Tær fyrstu vikurnar vórðu um 100 dripnar upp á vikuna, og so fóru tær at minnka, inntil eingin var eftir.

So var friður fyri teimum í nøkur ár. Men eina nátt undir fyrra veraldarbardaganum, meðan allir bátar vóru á útróðri, kom ein stórur motorbátur siglandi fram við landinum. Ein handilsmaður, ið væntaði ein bát at koma við varu til sín, fór upp at hyggja eftir bátinum. Hann legði inn móti Lendingarskerinum, og ein maður fór fram í stevnið, og so rýmdi báturin av stað aftur. Morgunin eftir var ein deyð mús funnin í einum neysti, og hon var vát og mundi helst vera sjókastað. Skjótt varð fólk aftur varigt við mýs, og so varð farið aftur at seta fellur; men hesa ferðina batti einki, tí nú vóru tær um alla oynna ísenn. Eingin var í iva um, at menninir á hesum bátinum høvdu gjørt eina ringa skálkagerð. Ein kona visti at siga, at henda royndin at sleppa músum á land her hevði verið ætlað fyrr, men hevði miseydnast vegna ókyrru.

Hetta var ein ljót gerð, og ein skaði, ið aldri bøtist aftur. Onkur kundi kanska havt hug at spurt, hvør ið tað var, ið hetta gjørdi. Í sovorðnum viðurskiftum plagdu tey gomlu at svara: ``Er hann nevndur, so er hann kendur.'' Vit fara eisini at svara við hesum somu orðum.

\nocite{*}

\printbibliography

\end{document}
